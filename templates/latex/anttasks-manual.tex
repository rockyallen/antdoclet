\documentclass[letterpaper,11pt,oneside]{article}

%% This document servers as a Template and introduction to your Ant
%% Tasks. At the end, it includes the output of AntDoclet, which will
%% provide all the reference documentation for your Ant tasks.


\usepackage{apache}
\usepackage{relsize}  % \smaller and \larger commands (to change fontsize relative to the current size).

% Important: The nameref, varioref and hyperref packages must be
% loaded in that order for proper functioning.
%
\usepackage{nameref} % Nameref (part of Hyperref package) for referencing sections by name (instead of number)
\usepackage{varioref} % for 'smart' references
\usepackage{calc}     % allow for algebraic calculations in length arguments

\lstset{escapechar=`}
\sloppy % allow for not-so-nice linebreaks
\usepackage[htt]{hyphenat} % Allow for TypeWriter text (ie \texttt{...}) to be hyphenated

\newcommand{\webLink}[2]{\href{#2}{#1}\footnote{\href{#2}{#2}}}



% specify the title/author
\title{AntDoclet Example}
\author{Fernando Dobladez\\\smaller\slshape{}fernando@dobladez.com}

% set the PDF Document Info
\hypersetup{
  pdftitle    = {Apache Ant Integration},
  pdfsubject  = {ANT},
  pdfkeywords = {ant,script,automation},
  pdfauthor   = {Fernando Dobladez},
}


%%%%%%%%%%%%%%%%%%%%%%%%%%%%%%%%%%%%%%%%%%%%%%%%%%%%%%%%%%%%%%%%%%%%%%%%%%%%%%
\begin{document}
\maketitle % generates the title page
\newpage
\begin{abstract}
  This is the reference documentation for MyAntTasks.

  Generated with \webLink{AntDoclet}{http://antdoclet.neuroning.com}.
\end{abstract}

\tableofcontents
\newpage


%%%
\section{Introduction}
These first sections of the document serve as an introduction for you
Ant Tasks documentation.

The rest of the sections are automatically generated by AntDoclet.

To modify this document template and this introductory content, edit
the \texttt{templates/example/latex/anttasks-manual.tex} file.

\subsection{Requirements}
You may want to clarify the version of Ant your Tasks depend on.
Example: Requires Ant 1.6.1 or newer, and it relies on the  \emph{namespace}
and \emph{antlib} features included in Ant version 1.6.

If you are providing an \emph{antlib}, you may want to explain how to
make available to Ant. Example: The provided antlib jar files need to
be specified using the -lib option:

\subsection{Using My Tasks}
You may provide a short snippet with a small but complete example of how
to use your Tasks:

\begin{lstlisting}
  <!-- This script uses my Ant Tasks -->
  <project name="Example" xmlns:mylib="antlib:package.to.my.lib">

    <!-- Include properties -->
    <property file="build.properties"/>

    <target name="do-it" description="Do something with my tasks">

      <mylib:myfirsttask argOne="true" >

      </mylib:myfirsttask>

    </target>

</project>
\end{lstlisting}


\vspace{2cm}
The following sections describe each of the provided Ant Tasks in
more detail, including all the attributes they accept and examples of
use.


%%
%% Include the output of AntDoclet (Ant Tasks reference documentation)
%%
\include{anttasks-reference}

\end{document}
